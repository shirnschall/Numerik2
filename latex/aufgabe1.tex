\documentclass[12pt,a4paper]{scrartcl}

\author{Rafael Dorigo, Sebastian Hirnschall}
%% (C) Hirnschall Sebastian, Rafael 2019
\date{\today}


\usepackage[
backend=biber,
style=authoryear-icomp,    % Zitierstil
isbn=false,                % ISBN nicht anzeigen, gleiches geht mit nahezu allen anderen Feldern
pagetracker=true,          % ebd. bei wiederholten Angaben (false=ausgeschaltet, page=Seite, spread=Doppelseite, true=automatisch)
maxbibnames=50,            % maximale Namen, die im Literaturverzeichnis angezeigt werden (ich wollte alle)
maxcitenames=3,            % maximale Namen, die im Text angezeigt werden, ab 4 wird u.a. nach den ersten Autor angezeigt
autocite=inline,           % regelt Aussehen für \autocite (inline=\parancite)
block=space,               % kleiner horizontaler Platz zwischen den Feldern
backref=true,              % Seiten anzeigen, auf denen die Referenz vorkommt
backrefstyle=three+,       % fasst Seiten zusammen, z.B. S. 2f, 6ff, 7-10
date=short                % Datumsformat
]{biblatex}

\addbibresource{./refs.bib}

\usepackage{longtable}
%\usepackage{hyperref}
\usepackage{amsmath}% http://ctan.org/pkg/amsmath
\usepackage[ngerman]{cleveref} %referenzen fur Abbildungen
\usepackage{graphicx}
\usepackage{listings}
\usepackage{esdiff}
\usepackage[utf8]{inputenc}
\usepackage[ngerman]{babel}
\usepackage[T1]{fontenc}
\usepackage{graphicx}
\usepackage{amssymb}
\usepackage{geometry}% http://ctan.org/pkg/geometry
\usepackage{amsthm}
\usepackage{tocloft}
\usepackage{framed}
\usepackage{mathtools}
\usepackage{color}
\usepackage{multirow}
\usepackage{textcomp}
\usepackage{float}
%\usepackage[dvipsnames]{xcolor}
\usepackage{mathtools}
\usepackage{caption}
\usepackage{subcaption}

%\usepackage[ruled,vlined]{algorithm2e}
\usepackage{algorithmicx}
\usepackage{algorithm}
\usepackage[noend]{algpseudocode}

\makeatletter
\def\BState{\State\hskip-\ALG@thistlm}
\makeatother

\usepackage[table,xcdraw,dvipsnames]{xcolor}

\usepackage{fancyvrb}


%\pagestyle{headings}

\setcounter{secnumdepth}{5}
\setcounter{tocdepth}{5}

%\pagestyle{headings}

\usepackage{fancyhdr}
\pagestyle{fancy}
%
\rhead{ }
%\rhead[re]{\textbf{\nouppercase{\leftmark}}}
\chead{}
\lhead{\leftmark}
%%
\lfoot{Rafael Dorigo, Sebastian Hirnschall}
\cfoot{}
\rfoot{\thepage}
%%
\renewcommand{\headrulewidth}{0.2pt}
\renewcommand{\footrulewidth}{0.2pt}
\newcommand{\R}{\mathbb{R}}


\fancypagestyle{general}
{
	\fancyhf{}
	\rhead{}
	%\rhead[re]{\textbf{\nouppercase{\leftmark}}}
	\chead{}
	\lhead{\leftmark}
	\lfoot{Rafael Dorigo, Sebastian Hirnschall}
	\cfoot{}
	\rfoot{\thepage}
}


%listings settings
\definecolor{mygreen}{rgb}{0,0.6,0}
\definecolor{mygray}{rgb}{0.5,0.5,0.5}
\definecolor{mymauve}{rgb}{0.58,0,0.82}
\definecolor{BackgroundGray}{rgb}{0.9,0.9,0.9}

\lstset{ %
	backgroundcolor=\color{BackgroundGray},   % choose the background color; you must add \usepackage{color} or \usepackage{xcolor}
	basicstyle=\footnotesize,        % the size of the fonts that are used for the code
	breakatwhitespace=false,         % sets if automatic breaks should only happen at whitespace
	breaklines=true,                 % sets automatic line breaking
	captionpos=b,                    % sets the caption-position to bottom
	commentstyle=\color{mygreen},    % comment style
	deletekeywords={...},            % if you want to delete keywords from the given language
	escapeinside={\%*}{*)},          % if you want to add LaTeX within your code
	extendedchars=true,              % lets you use non-ASCII characters; for 8-bits encodings only, does not work with UTF-8
	frame=single,	                   % adds a frame around the code
	keepspaces=true,                 % keeps spaces in text, useful for keeping indentation of code (possibly needs columns=flexible)
	keywordstyle=\color{blue},       % keyword style
	language=C,                 	   % the language of the code
	otherkeywords={*,...},           % if you want to add more keywords to the set
	numbers=left,                    % where to put the line-numbers; possible values are (none, left, right)
	numbersep=5pt,                   % how far the line-numbers are from the code
	numberstyle=\tiny\color{mygray}, % the style that is used for the line-numbers
	rulecolor=\color{mygray},         % if not set, the frame-color may be changed on line-breaks within not-black text (e.g. comments (green here))
	showspaces=false,                % show spaces everywhere adding particular underscores; it overrides 'showstringspaces'
	showstringspaces=false,          % underline spaces within strings only
	showtabs=false,                  % show tabs within strings adding particular underscores
	stepnumber=2,                    % the step between two line-numbers. If it's 1, each line will be numbered
	stringstyle=\color{mymauve},     % string literal style
	tabsize=2,	                   % sets default tabsize to 2 spaces
	title=\lstname,                   % show the filename of files included with \lstinputlisting; also try caption instead of title
	emph={int,unsigned,long,vector,char,string},
	emphstyle={\color{ForestGreen}}
}

\Crefname{lstlisting}{Listing}{Listing}


%italic quotes
\newenvironment{italicquotes}
{\begin{quote}\itshape}
	{\end{quote}}


%tableofcontents font
%\renewcommand{\cftchapfont}{\scshape}
\renewcommand{\cftsecfont}{\bfseries}
\addtokomafont{disposition}{\rmfamily}

\newcommand{\spar}{\par\vspace{10pt}\noindent}
\newcommand{\Mod}[1]{\ (\text{mod}\ #1)}


\usepackage{twoopt}
\newcommandtwoopt{\img}[4][0.5cm][0.7]{
	\begin{figure}[!h]
		\vspace{#1}
		\centering
		\includegraphics[width=#2\textwidth]{#3}
		\caption{#4} %\footnotemark}
		\label{fig:#3}
	\end{figure}
	%\footnotetext{#5}
}


\numberwithin{equation}{section} 
%\makeatletter
%\@addtoreset{equation}{section}
%\makeatother


%\newtheorem{theorem}{Theorem}[section]
%\newtheorem{lemma}[theorem]{Lemma}
%\newtheorem{proposition}[theorem]{Proposition}
%\newtheorem{corollary}[theorem]{Corollary}

\newcounter{myalgctr}

\newenvironment{mydef}{%      define a custom environment
	\par\noindent%         create a vertical offset to previous material
	\refstepcounter{myalgctr}% increment the environment's counter
	\textsc{\textbf{Definition} \themyalgctr}% or \textbf, \textit, ...
	\newline
}{\par\bigskip}  %          create a vertical offset to following material
\numberwithin{myalgctr}{section}

\crefname{myalgctr}{Definition}{Definitionen}

%theorem
\newcounter{mytheoremctr}

\newenvironment{mytheorem}{%      define a custom environment
	\refstepcounter{mylemmactr}% increment the environment's counter
	\refstepcounter{mykorollarctr}
	\refstepcounter{mybeispielctr}% increment the environment's counter
	\refstepcounter{mytheoremctr}
	\par \noindent%         create a vertical offset to previous material
	\textsc{\textbf{Satz} \themytheoremctr}% or \textbf, \textit, ...
	\newline\noindent
}{\par\bigskip}  %          create a vertical offset to following material
\numberwithin{mytheoremctr}{subsection}

\crefname{mytheoremctr}{Satz}{Satz}

%korollar
\newcounter{mykorollarctr}

\newenvironment{mykorollar}{%      define a custom environment
	\refstepcounter{mylemmactr}% increment the environment's counter
	\refstepcounter{mytheoremctr}
	\refstepcounter{mybeispielctr}% increment the environment's counter
	\refstepcounter{mykorollarctr}
	\par\noindent%         create a vertical offset to previous material
	\textsc{\textbf{Korollar} \themykorollarctr}% or \textbf, \textit, ...
	\newline\noindent
}{\par\bigskip}  %          create a vertical offset to following material
\numberwithin{mykorollarctr}{subsection}

\crefname{mykorollarctr}{Korollar}{Korollar}

%lemma
\newcounter{mylemmactr}

\newenvironment{mylemma}{%      define a custom environment
	\refstepcounter{mykorollarctr}
	\refstepcounter{mytheoremctr}
	\refstepcounter{mybeispielctr}% increment the environment's counter
	\refstepcounter{mylemmactr}% increment the environment's counter
	\par\noindent%         create a vertical offset to previous material
	\textsc{\textbf{Lemma} \themylemmactr}% or \textbf, \textit, ...
	\newline\noindent
}{\par\bigskip}  %          create a vertical offset to following material
\numberwithin{mylemmactr}{subsection}

\crefname{mylemmactr}{Lemma}{Lemma}

\newcounter{mybeispielctr}

%beispiel
\newenvironment{mybeispiel}{%      define a custom environment
	\refstepcounter{mykorollarctr}
	\refstepcounter{mytheoremctr}
	\refstepcounter{mylemmactr}% increment the environment's counter
	\refstepcounter{mybeispielctr}% increment the environment's counter
	\par\noindent%         create a vertical offset to previous material
	\textsc{\textbf{Beispiel} \themybeispielctr}% or \textbf, \textit, ...
	\newline\noindent
}{\par\bigskip}  %          create a vertical offset to following material
\numberwithin{mybeispielctr}{subsection}

\crefname{mybeispielctr}{Beispiel}{Beispiel}

\newenvironment{myproof}{%      define a custom environment
	\bigskip\noindent%         create a vertical offset to previous material
	\textsc{\textbf{\\Beweis\\}}% or \textbf, \textit, ...
	\indent
}{\qed\par\bigskip}  %          create a vertical offset to following material

\newenvironment{bemerkung}{%      define a custom environment
	\bigskip\noindent%         create a vertical offset to previous material
	\textsc{\textbf{\\Bemerkung.}}% or \textbf, \textit, ...
	\indent
}{\par\bigskip}  %          create a vertical offset to following material

\newcommand{\mpar}[1]{\paragraph*{#1}\mbox{}\par}
\newcommand\norm[1]{\left\lVert#1\right\rVert}
\DeclarePairedDelimiter{\abs}{\lvert}{\rvert}


\pagestyle{fancy}
\fancypagestyle{firststyle}
{
	\fancyhf{}
	\rhead{}
	%\rhead[re]{\textbf{\nouppercase{\leftmark}}}
	\chead{}
	\lhead{\leftmark}
	\lfoot{Rafael Dorigo, Sebastian Hirnschall}
	\cfoot{}
	\rfoot{\thepage}
}

\crefname{section}{Abschnitt}{Abschnitt}

\setlength\parindent{0pt}
\begin{document}
	\newgeometry{bottom=1cm,top=1cm,left=1cm,right=1cm}
	\begin{titlepage}
		\begin{flushleft}
				\includegraphics[width=.4\linewidth]{tuwien.png}
		\end{flushleft}	
		\centering
		
		
		\vspace{5cm}
		{\huge\bfseries Lineare Gleichungssysteme und \\Dünn besetzte Matrizen\par}
		\vspace{2cm}
		{\Large\itshape Rafael Dorigo\\Sebastian Hirnschall\par}
		\vspace{1cm}
		{\large\itshape Betreut von:\\Markus Wess Dipl.-Ing.\par}
		\vspace{1cm}
		\begin{figure}[!h]
			\vspace{0cm}
			\centering
			\includegraphics[width=.4\linewidth]{titelbild2.png}
		\end{figure}
		
		\vfill
		
		% Bottom of the page
		{\today\par}
	\end{titlepage}
	\restoregeometry
	
	\thispagestyle{firststyle}
	
	\newpage\noindent
	{\LARGE \bfseries Abstract}
	\newline
	\par\noindent
	
	\thispagestyle{firststyle}
	
	\newpage
	\tableofcontents
	\thispagestyle{general}
	\newpage

	\section{Einleitung}
	\newpage
	
	\section{CG-Verfahren (Conjugate Gradients)}
	
	\begin{algorithm}[H]
		\textbf{Input:} Sei $A\in$ $\mathbb{R}^{n\times n}, b, x_0 \in \mathbb{R}^{n}$ und eine Toleranz $\tau > 0$
		\begin{algorithmic}[1]
			\State $r_0 = b - Ax_0$
			\State $d_0 = r_0$
			\State $t = 0$
			\While{$ \norm{r_t} > \tau $}
			\State $\alpha_t = \frac{r_t^{T}r_t}{d_t^{T}Ad_t}$
			\State $x_{t+1} = x_{t} + \alpha_t d_t$
			\State $r_{t+1} = r_t - \alpha_t Ad_t$
			\State $\beta_t = \frac{r_{t+1}^{T}r_{t+1}}{r_t^{T}r_t}$
			\State $d_{t+1} = r_{t+1} + \beta_td_t$
			\State $t = t + 1$
			\EndWhile
		\end{algorithmic}
		\textbf{Output:} Näherung $x_t$ an $x = A^{-1}b$ mit $\norm{Ax_t-b} < \tau$
		
		\caption{CG-Verfahren} \label{alg:cg}
	\end{algorithm}
	
	Nun stellt sich die Frage, ob \cref{alg:cg} äquivalent zum Algorithmus 8.10 \autocite[101]{skript} ist und welches bevorzugt wird.
	
	\newpage
	
	Da man beim lösen linearer Gleichungssysteme oft mit großen, dünn besetzten Matrizen arbeitet, ist es nicht sinnvoll die ganze Matrix im Rechner zu speichern. Deshalb bietet sich das sogenannte \textit{"compressed spare row"} Format, welches nur die Einträge der Matrix ungleich null betrachtet und mit diesen rechnet. Bei der Implementierung dieses Formats werden anstelle aller Einträge $A_{i,j}, i,j = 1,\ldots,n$ einer Matrix $A\in$ $\mathbb{R}^{n\times n}$ ein Vektor $v\in\mathbb{R}^{m}$ aller Einträge ungleich Null, ein Vektor $J\in\mathbb{N}_0^{m}$ von Spaltenindizes und ein Vektor $I\in\mathbb{N}_0^{n+1}$ gespeichert.
	
	Die $i-$te Zeile von A ist gegeben durch
	\begin{align*}
		A_{i,j} = 
		\begin{cases}
			\textit{$v_{k(j)}$},&\quad\textit{falls $j \in \{J_{I_i}, J_{I_{i}} + 1, \ldots, J_{I_{i+1}} - 1$}\}\\
			\textit{0},&\quad\textit{sonst}\\
		\end{cases}
	\end{align*} 
	wobei $J_{k_{(j)}} = j$.\\
	
	Eine Möglichkeit das Format zu implementieren ist wie folgt:
	
	hier code
	
	Die spektrale Konditionszahl $cond(A) = \frac{\lambda_{max}(A)}{\lambda_{min}(A)}$ definiert die Konvergenzgeschwindigkeit des CG-Verfahrens. Durch lösen des vorkonditionierten Systems
	\begin{align*}
		D^{-1}AD^{-T}y = D^{-1}b
	\end{align*}
	kann die Konvergenz beschleunigt werden und man bekommt eine Lösung x der Form 
	\begin{align*}
		x = D^{-T}y
	\end{align*}
	Man wählt die Matrix D so, dass für beliebige $z \in \mathbb{R}^{n}$ der Vektor $D^{-T}D^{-1}z$ einfach zu berechnen ist und zugleich, dass $cond(D^{-1}AD^{-T}) < cond(A)$ gilt.
	\begin{algorithm}[H]
		\textbf{Input:} Sei A$\in$ $\mathbb{R}^{n\times n}, x_0 \in \mathbb{R}^{n}, P:=DD^{T}$ und eine Toleranz $\tau > 0$
		\begin{algorithmic}[1]
			\State $r_0 = b - Ax_0$
			\State $z_0 = P^{-1}r_0$
			\State $d_0 = r_0$
			\State $t = 0$
			\While{$ \norm{r_t} > \tau $}
			\State $\alpha_t = \frac{r_t^{T}z_t}{d_t^{T}Ad_t}$
			\State $x_{t+1} = x_{t} + \alpha_t d_t$
			\State $r_{t+1} = r_t - \alpha_t Ad_t$
			\State $z_{t+1} = P^{-1}r_{t+1}$
			\State $\beta_t = \frac{z_{t+1}^{T}r_{t+1}}{z_tr_t}$
			\State $d_{t+1} = r_{t+1} + \beta_td_t$
			\State $t = t + 1$
			\EndWhile
		\end{algorithmic}
		\textbf{Output:} Näherung $x_t$ an $x = A^{-1}b$ mit $\norm{Ax_t-b} < \tau$
		
		\caption{Vorkonditionierte CG-Verfahren} \label{alg:vcg}
	\end{algorithm}
	
	\newpage
	\appendix
	\section{Verwendete Klassen}
	\subsection{Vollbesetzte Matrix}
	\lstinputlisting[language=c,  caption=densematrix.h]{../code/aufgabe1/densematrix.h}
	
	\newpage
	\printbibliography
	\listoffigures
	\thispagestyle{firststyle}
	
\end{document}

